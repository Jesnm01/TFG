\chapter{Tabla comparativa del set completo de moléculas}

\label{apend:tabla_intro_grande}


\begin{landscape}
% \begin{table}
% \begin{tabularx}{24cm}{|X|X|X|X|X|X|X|}
%    \hline
%     & \textbf{IEEE} & \textbf{CODATA} & \textbf{ACM} & \textbf{Springer Verlag} & \textbf{ELSEVIER} & \textbf{IOS PRESS} \\ 
%     \hline
%     Journal & Journal Transactions on Knowledge and Data Engineering & Data Science Journal & Journal of Data and Information Quality & International Journal of Data Science & Computional Statistics and Data Analysis & Data Science Journal \\
%     \hline
%     Organisation bzw. Verlag & Organisation & Organisation & Organisation & Verlag & Verlag & Verlag \\
%     \hline
%     Mitglieder bzw. Mitarbeiter & 400.000 Mitglieder & - & 78.000 Mitglieder & 15.323 (2016) Mitarbeiter & 30.500 (2011) Mitarbeiter & - \\
%     \hline
%     Editoren & Editor-in-Chief Xuemin Lin Editors-in-Chief Lei Chen & Editor-in-Chief Sarah Callaghan & - & Editor-in-Chief Longbing Cao & Co-Editors A.M. Colubi E.J. Kontoghiorghes B.U. Park & Editors-in-Chief Michel Dumontier Tobias Kuhn \\
%     \hline
%     Links & \url{http://ieeexplore.ieee.org/xpl/RecentIssue.jsp?punumber=69} & \url{https://datascience.codata.org/} & \url{https://dl.acm.org/citation.cfm?id=J1191} & \url{http://www.springer.com/computer/database+management+\%26+information+retrieval/journal/41060} & \url{https://www.journals.elsevier.com/computational-statistics-and-data-analysis/} & \url{https://www.iospress.nl/journal/data-science/} \\
%     \hline
%     Erstausgabe Jahr & 1989 & 2002 & 2009 & 2016 & 1983 & 2017 \\
%     \hline
% \end{tabularx}
% \caption{Übersicht der Konkurrenten}
% \end{table}


% \begin{table}[h!]
% \small
% \centering
%     \begin{tabular}{m{3cm}>{\centering\arraybackslash}m{6cm}m{5cm}m{4cm}m{4cm}}
%         \hline
%         \textbf{Nombre} & \textbf{SMILES SA} & \textbf{SMILES SF} & \textbf{Imagen SA} & \textbf{Imagen SF} \\
%         \hline
%         & & [Au+]([CH3-])[P](C=1C=CC=CC1) (C=2C=CC=CC2)C=3C=CC=CC3 &  & \includegraphics[width=2.2cm]{imagenes/sciFinder/Methyl(triphenylphosphine)gold(I).png} \\
%         \hline
%         & & [Cl-][Pd+2]123([Cl-]) [CH]=4CC[CH]3=[CH]2CC[CH]41 & & \includegraphics[width=2.2cm]{imagenes/sciFinder/Dichloro(1,5-cyclooctadiene)palladium(II).png} \\
%         \hline
%         & & [Cl-][Au+][P](C)(C)C & & \includegraphics[width=2.2cm]{imagenes/sciFinder/Chloro(trimethylphosphine)gold(I).png} \\
%         \hline
%         & & [Cl-][Au+][P](C=1C=CC=CC1C=2C=CC=CC2) (C(C)(C)C)C(C)(C)C & & \includegraphics[width=2.2cm]{imagenes/sciFinder/Chloro[(1,1-biphenyl-2-yl)di-tert-butylphosphine]gold(I).png} \\
%         \hline
%         & & O\#C[Fe+2]1234([I-])(C\#O)[CH]=5[CH]4 =[CH]3[CH-]2[CH]51 & & \includegraphics[width=2.2cm]{imagenes/sciFinder/Dicarbonylcyclopentadienyliodoiron(II).png} \\
%         \hline




%         & & O\#C[Fe+2]1234([I-])(C\#O)[CH]=5[CH]4 =[CH]3[CH-]2[CH]51 & & \includegraphics[width=2.2cm]{imagenes/sciFinder/Dicarbonylcyclopentadienyliodoiron(II).png} \\
%         \hline
%         & & O\#C[Fe+2]1234([I-])(C\#O)[CH]=5[CH]4 =[CH]3[CH-]2[CH]51 & & \includegraphics[width=2.2cm]{imagenes/sciFinder/Dicarbonylcyclopentadienyliodoiron(II).png} \\
%         \hline
%         & & O\#C[Fe+2]1234([I-])(C\#O)[CH]=5[CH]4 =[CH]3[CH-]2[CH]51 & & \includegraphics[width=2.2cm]{imagenes/sciFinder/Dicarbonylcyclopentadienyliodoiron(II).png} \\
%         \hline
%         & & O\#C[Fe+2]1234([I-])(C\#O)[CH]=5[CH]4 =[CH]3[CH-]2[CH]51 & & \includegraphics[width=2.2cm]{imagenes/sciFinder/Dicarbonylcyclopentadienyliodoiron(II).png} \\
%         \hline
%         & & O\#C[Fe+2]1234([I-])(C\#O)[CH]=5[CH]4 =[CH]3[CH-]2[CH]51 & & \includegraphics[width=2.2cm]{imagenes/sciFinder/Dicarbonylcyclopentadienyliodoiron(II).png} \\
%         \hline

%         \end{tabular}
%     \caption{Tabla extendida para el set de datos de 30 moléculas. Contiene el nombre del compuesto (uno de los varios sinónimos según el formato de nomenclatura de la IUPAC), la cadena SMILES extraída de Sigma-Aldrich (SA), la cadena SMILES extraída de SciFinder (SF), y las imágenes de las respectivas bases de datos (SA y SF)}
%     \label{tabla:tabla_grande_intro_label}
% \end{table}




\begin{longtable}{m{7cm}m{8cm}m{2.3cm}m{2.3cm}}
\caption{Tabla extendida para el set de datos de 30 moléculas. Contiene la cadena SMILES extraída de Sigma-Aldrich (SA), la cadena SMILES extraída de SciFinder (SF), y las imágenes de las respectivas bases de datos (SA y SF)}\\
\hline
\textbf{SMILES SA} & \textbf{SMILES SF} & \textbf{Imagen SA} & \textbf{Imagen SF} \\ \hline
\endfirsthead

\multicolumn{4}{c}%
{{\bfseries \tablename\ \thetable{} -- Continuación de la tabla en la página siguiente}} \\
\hline
\textbf{SMILES SA} & \textbf{SMILES SF} & \textbf{Imagen SA} & \textbf{Imagen SF} \\ \hline
\endhead

\hline \multicolumn{4}{r}{{Continúa en la siguiente página}} \\
\endfoot

\hline
\endlastfoot

% Compuesto 2
 C[Au].c1ccc(cc1)P(c2ccccc2) c3ccccc3 & 
 [Au+]([CH3-])[P](C=1C=CC=CC1) (C=2C=CC=CC2)C=3C=CC=CC3 & 
 \includegraphics[width=2.2cm]{imagenes/sigmaAldrich/Methyl(triphenylphosphine)gold(I).png} & 
 \includegraphics[width=2.2cm]{imagenes/sciFinder/pdf/Methyl(triphenylphosphine)gold(I).pdf} \\
\hline

% Compuesto 3
 Br[Pd]Br.c1ccc(cc1) P(c2ccccc2)c3ccccc3.c4ccc(cc4) P(c5ccccc5)c6ccccc6 & 
 [Br-][Pd+2]([Br-])([P](C=1C= CC=CC1)(C=2C=CC=CC2) C=3C=CC=CC3)[P](C=4C=CC=CC4) (C=5C=CC= CC5)C=6C=CC=CC6 & 
 \includegraphics[width=2.2cm]{imagenes/sigmaAldrich/trans-Dibromobis(triphenylphosphine)palladium(II).png} & 
 \includegraphics[width=2.2cm]{imagenes/sciFinder/pdf/trans-Dibromobis(triphenylphosphine)palladium(II).pdf} \\
\hline

% Compuesto 4
 Cl[Pd]Cl.C1CC=CCCC=C1 & 
 [Cl-][Pd+2]123([Cl-]) [CH]=4CC[CH]3=[CH]2CC[CH]41 & 
 \includegraphics[width=2.2cm]{imagenes/sigmaAldrich/Dichloro(1,5-cyclooctadiene)palladium(II).png} & 
 \includegraphics[width=2.2cm]{imagenes/sciFinder/pdf/Dichloro(1,5-cyclooctadiene)palladium(II).pdf} \\
\hline

% Compuesto 5
 C1C[C@@H]2C[C@H]1CC2PC3C [C@@H]4CC[C@H]3C4.CN(C)c5ccccc5-c6ccccc6[Pd]Cl & 
 [Cl-][Pd+2]1([C-]=2C=CC=CC2C=3C =CC=CC3[N]1(C)C)[PH] (C4CC5CCC4C5)C6CC7CCC6C7 & 
 \includegraphics[width=2.1cm]{imagenes/sigmaAldrich/SK-CC 01A.jpeg} & 
 \includegraphics[width=2.2cm, height=2.1cm]{imagenes/sciFinder/pdf/SK-CC 01A.pdf} \\
\hline

% Compuesto 6
 C\textbackslash C(=N/O)c1ccc(O)cc1[Pd]Cl .C\textbackslash C(=N/O)c2ccc(O)cc2[Pd]Cl & 
 OC=1C=CC=2C(=[N](O)[Pd+2]3 ([Cl-][Pd+2]4([Cl-]3)[C-]=5 C=C(O)C=CC5C(=[N]4O)C)[C-]2C1)C & 
 \includegraphics[width=2.2cm]{imagenes/sigmaAldrich/Bis[µ-chloro[5-hydroxy-2-[1-(hydroxyimino)ethyl]phenyl]palladium].jpeg} & 
 \includegraphics[width=2.2cm]{imagenes/sciFinder/pdf/Bis[µ-chloro[5-hydroxy-2-[1-(hydroxyimino)ethyl]phenyl]palladium].pdf} \\
\hline


% Compuesto 7
 No se encontró el compuesto en Sigma-Aldrich & 
 FC=1C(Cl)=C(F)[C-](=C(F)C1Cl)[Pd+2] ([I-])([As](C=2C=CC=CC2)(C=3C=CC=C C3)C=4C=CC=CC4)[As](C=5C=CC=CC5) (C=6C=CC=CC6)C=7C=CC=CC7 & 
 & 
 \includegraphics[width=2.5cm]{imagenes/sciFinder/pdf/(SP-4-3)-(3,5-Dichloro-2,4,6-trifluorophenyl)iodobis(triphenylarsine)palladium.pdf} \\
\hline


% Compuesto 8
 No se encontró el compuesto en Sigma-Aldrich & 
 O=S(=O)([NH-][Pd+4]12([F-])([C-]=3C=CC=CC3C(C)(C)[CH2-]1) [N]=4C=CC=CC4C=5C=CC=C[N] 52)C6=CC=C(C=C6)C & 
 & 
 \includegraphics[width=2.2cm]{imagenes/sciFinder/pdf/Palladium, (2,2-bipyridine-κN1,κN1)[(2,2-dimethyl-1,2-ethanediyl)-1,2-phenylene]fluoro(4-methylbenzenesulfonamidato-κN)-, (OC-6-35).pdf} \\
\hline


% Compuesto 9
 Br[Ni]Br.COCCOC & 
 [Br-][Ni+2]1([Br-])O(C)CCO1C & 
 \includegraphics[width=2.2cm]{imagenes/sigmaAldrich/Nickel(II) bromide ethylene glycol dimethyl ether complex.png} & 
 \includegraphics[width=2.2cm]{imagenes/sciFinder/pdf/Dibromo(1,2-dimethoxyethane)nickel(II).pdf} \\
\hline


% Compuesto 10
 Cl[Ru](Cl)(C\#[O])(C\#[O])([PH](c1ccccc1) (c2ccccc2)c3ccccc3)[PH](c4ccccc4) (c5ccccc5)c6ccccc6 & 
 O\#C[Ru+2]([Cl-])([Cl-])(C\#O)([P](C=1C=CC=CC1) (C=2C=CC=CC2)C=3C=CC=CC3)[P] (C=4C=CC=CC4)(C=5C=CC=CC5) C=6C=CC=CC6 & 
 \includegraphics[width=2.2cm]{imagenes/sigmaAldrich/Bis(triphenylphosphine)ruthenium(II) dicarbonyl chloride.jpeg} & 
 \includegraphics[width=2.2cm]{imagenes/sciFinder/pdf/Bis(triphenylphosphine)ruthenium(II) dicarbonyl chloride.pdf} \\
\hline

% Compuesto 11
 Cl[Au].CP(C)C & 
 [Cl-][Au+][P](C)(C)C & 
 \includegraphics[width=2.2cm]{imagenes/sigmaAldrich/Chloro(trimethylphosphine)gold(I).png} & 
 \includegraphics[width=2.2cm]{imagenes/sciFinder/pdf/Chloro(trimethylphosphine)gold(I).pdf} \\
\hline


% Compuesto 12
 Cl[Au].FC(F)(F)c1ccc(cc1)P(c2ccc (cc2)C(F)(F)F)c3ccc(cc3)C(F)(F)F & 
 FC(F)(F)C1=CC=C(C=C1)[P] ([Au+][Cl-])(C2=CC=C(C=C2)C(F) (F)F)C3=CC=C(C=C3)C(F)(F)F & 
 \includegraphics[width=2.2cm]{imagenes/sigmaAldrich/Chloro[tris(para-trifluoromethylphenyl)phosphine]gold(I).png} & 
 \includegraphics[width=2.2cm]{imagenes/sciFinder/pdf/Chloro[tris(para-trifluoromethylphenyl)phosphine]gold(I).pdf} \\
\hline



% Compuesto 13
 Cl[Au].CSC & 
 [Cl-][Au+][S](C)C & 
 \includegraphics[width=2.2cm]{imagenes/sigmaAldrich/Chloro(dimethylsulfide)gold(I).png} & 
 \includegraphics[width=2.2cm]{imagenes/sciFinder/pdf/Chloro(dimethylsulfide)gold(I).pdf} \\
\hline



% Compuesto 14
 Cl[Au].CP(c1ccccc1)c2ccccc2 & 
 [Cl-][Au+][P](C=1C=CC=CC1) (C=2C=CC=CC2)C & 
 \includegraphics[width=2.1cm, height=1.5cm]{imagenes/sigmaAldrich/Chloro(methyldiphenylphosphine)gold(I).jpeg} & 
 \includegraphics[width=2.2cm]{imagenes/sciFinder/pdf/Chloro(methyldiphenylphosphine)gold(I).pdf} \\
\hline




% Compuesto 15
 Cl[Au].Cc1ccccc1P(c2ccccc2)c3ccccc3 & 
 [Cl-][Au+][P](C=1C=CC=CC1) (C=2C=CC=CC2)C=3C=CC=CC3C & 
 \includegraphics[width=2.2cm]{imagenes/sigmaAldrich/Chloro[diphenyl(o-tolyl)phosphine]gold(I).jpeg} & 
 \includegraphics[width=2.2cm]{imagenes/sciFinder/pdf/Chloro[diphenyl(o-tolyl)phosphine]gold(I).pdf} \\
\hline


% Compuesto 16
 Cl[Au].CC(C)(C)P(c1ccccc1-c2ccccc2)C(C)(C)C & 
 [Cl-][Au+][P](C=1C=CC=CC1C=2C =CC=CC2)(C(C)(C)C)C(C)(C)C & 
 \includegraphics[width=2.2cm]{imagenes/sigmaAldrich/Chloro[(1,1-biphenyl-2-yl)di-tert-butylphosphine]gold(I).png} & 
 \includegraphics[width=2.2cm]{imagenes/sciFinder/pdf/Chloro[(1,1-biphenyl-2-yl)di-tert-butylphosphine]gold(I).pdf} \\
\hline



% Compuesto 17
 [Au+].CC\#N.F[Sb-](F)(F)(F)(F)F. CC(C)(C)P(c1ccccc1-c2ccccc2)C(C)(C)C & 
 [F-][Sb+5]([F-])([F-])([F-])([F-])[F-]. C(\#[N][Au+][P](C=1C=CC=CC1C= 2C=CC=CC2)(C(C)(C)C)C(C)(C)C)C & 
 \includegraphics[width=2.2cm]{imagenes/sigmaAldrich/(Acetonitrile)[(2-biphenyl)di-tert-butylphosphine]gold(I) hexafluoroantimonate.jpeg} & 
 \includegraphics[width=2.2cm]{imagenes/sciFinder/pdf/(Acetonitrile)[(2-biphenyl)di-tert-butylphosphine]gold(I) hexafluoroantimonate [1compuesto].pdf} \\
  &  &  & 
 \includegraphics[width=2.2cm]{imagenes/sciFinder/pdf/(Acetonitrile)[(2-biphenyl)di-tert-butylphosphine]gold(I) hexafluoroantimonate [2compuesto].pdf} \\
\hline



% Compuesto 18
 CC(C)c1cc(C(C)C)c(c(c1)C(C)C)-c2 ccccc2[PH]([Au]Cl)(C(C)(C)C)C(C)(C)C & 
 [Cl-][Au+][P](C=1C=CC=CC1C=2C (=CC(=CC2C(C)C)C(C)C)C(C)C)(C(C) (C)C)C(C)(C)C & 
 \includegraphics[width=2.2cm]{imagenes/sigmaAldrich/Chloro[2-di-tert-butyl(2,4,6-triisopropylbiphenyl)phosphine] gold(I).jpeg} & 
 \includegraphics[width=2.3cm]{imagenes/sciFinder/pdf/Chloro[2-di-tert-butyl(2,4,6-triisopropylbiphenyl)phosphine] gold(I).pdf} \\
\hline



% Compuesto 19
 Cl[Au].CC(C)c1cc(C(C)C)c(c(c1)C (C)C)-c2ccccc2P(C3CCCCC3)C4CCCCC4 & 
 [Cl-][Au+][P](C=1C=CC=CC1C=2C (=CC(=CC2C(C)C)C(C)C)C(C)C)(C3C CCCC3)C4CCCCC4 & 
 \includegraphics[width=2.2cm]{imagenes/sigmaAldrich/Chloro[2-dicyclohexyl(2,4,6-trisopropylbiphenyl)phosphine]gold(I).png} & 
 \includegraphics[width=2.2cm]{imagenes/sciFinder/pdf/Chloro[2-dicyclohexyl(2,4,6-trisopropylbiphenyl)phosphine]gold(I).pdf} \\
\hline



% Compuesto 20
 CC(C)OC(C=CC=C1OC(C)C)=C1C2 =CC(P(C3CCCCC3)C4=CC(C5=C(OC (C)C)C=CC=C5OC(C)C)=CC=C4) =CC=C2.[Au]Cl & 
 [Cl-][Au+][P](C=1C=CC=CC1C2=C(OC (C)C)C=CC=C2OC(C)C)(C=3C=CC=CC3 C4=C(OC(C)C)C=CC=C4OC(C)C)C5CCCCC5 & 
 \includegraphics[width=2.2cm]{imagenes/sigmaAldrich/pdf/BisPhePhos XD gold(I) chloride.pdf} & 
 \includegraphics[width=2.2cm]{imagenes/sciFinder/pdf/BisPhePhos XD gold(I) chloride.pdf} \\
\hline



% Compuesto 21
 Cl[Au].CN(C)c1ccccc1P(C23C C4CC(CC(C4)C2)C3)C56CC7CC(C C(C7)C5)C6 & 
 [Cl-][Au+][P](C=1C=CC=CC1N (C)C)(C23CC4CC(CC(C4)C2)C3) C56CC7CC(CC(C7)C5)C6 & 
 \includegraphics[width=2.2cm]{imagenes/sigmaAldrich/Chloro[di(1-adamantyl)-2-dimethylaminophenylphosphine]gold(I).png} & 
 \includegraphics[width=2.2cm]{imagenes/sciFinder/pdf/Chloro[di(1-adamantyl)-2-dimethylaminophenylphosphine]gold(I).pdf} \\
\hline





% Compuesto 22
 Cl[Au].Cl[Au].C(CP(c1ccccc1) c2ccccc2)P(c3ccccc3)c4ccccc4 & 
 [Cl-][Au+][P](C=1C=CC=CC1) (C=2C=CC=CC2)CC[P]([Au+][Cl-]) (C=3C=CC=CC3)C=4C=CC=CC4 & 
 \includegraphics[width=2.2cm]{imagenes/sigmaAldrich/Dichloro(DPPE)digold(I).jpeg} & 
 \includegraphics[width=2.2cm]{imagenes/sciFinder/pdf/Dichloro(DPPE)digold(I).pdf} \\
\hline


% Compuesto 23
 Cl[Au].Cl[Au].P(C1=CC=CC=C1) (C2=C(C3=C(C=CC4=C3C=CC=C4) P(C5=CC=CC=C5)C6=CC=CC=C6)C7 =C(C=CC=C7)C=C2)C8=CC=CC=C8 & 
 [Cl-][Au+][P](C=1C=CC=CC1) (C=2C=CC=CC2)C3=CC=C4C=CC=CC4 =C3C=5C=6C=CC=CC6C=CC5[P] ([Au+][Cl-])(C=7C=CC=CC7)C=8C=CC=CC8 & 
 \includegraphics[width=2.2cm]{imagenes/sigmaAldrich/Dichloro[(±)−BINAP]digold(I).png} & 
 \includegraphics[width=2.2cm]{imagenes/sciFinder/pdf/Dichloro[(±)−BINAP]digold(I).pdf} \\
\hline


% Compuesto 24
 [Fe].Cl[Au].Cl[Au].[CH]1[CH] [CH][C]([CH]1)P(c2ccccc2)c3 ccccc3.[CH]4[CH][CH][C]([CH]4) P(c5ccccc5)c6ccccc6 & 
 [Cl-][Au+][P](C=1C=CC=CC1)(C=2C=CC =CC2)[C-]34[CH]5=[CH]6[CH]7=[CH]3[Fe+2] 6789\%10\%1154[CH]=\%12[CH]\%11=[CH]\%10 [C-]9([CH]\%128)[P]([Au+][Cl-])(C=\%13 C=CC=CC\%13)C=\%14C=CC=CC\%14 & 
 \includegraphics[width=2.5cm]{imagenes/sigmaAldrich/pdf/Bis(chlorogold(I)) [1,1-bis(diphenylphosphino)ferrocene.png} & 
 \includegraphics[width=2.2cm]{imagenes/sciFinder/pdf/Bis(chlorogold(I)) [1,1-bis(diphenylphosphino)ferrocene].pdf} \\
\hline


% Compuesto 25
 Cl[Au].Cc1cc(C)c(N2[C]N(C=C2) c3c(C)cc(C)cc3C)c(C)c1 & 
 Cl[Au]=C1N(C=CN1C=2C(=CC(=CC2 C)C)C)C=3C(=CC(=CC3C)C)C & 
 \includegraphics[width=2.2cm]{imagenes/sigmaAldrich/pdf/[(IMes)AuCl].pdf} & 
 \includegraphics[width=2.2cm]{imagenes/sciFinder/pdf/[(IMes)AuCl].pdf} \\
\hline


% Compuesto 26
 CC(C)c1cccc(C(C)C)c1N2C=CN(C2 [Au]Cl)c3c(cccc3C(C)C)C(C)C & 
 Cl[Au]=C1N(C=CN1C=2C(=CC=CC2C (C)C)C(C)C)C=3C(=CC=CC3C(C)C)C(C)C & 
 \includegraphics[width=2.2cm]{imagenes/sigmaAldrich/[(IPr)AuCl].png} & 
 \includegraphics[width=2.2cm]{imagenes/sciFinder/pdf/[(IPr)AuCl].pdf} \\
\hline

% Compuesto 27
 CC(C)c1cccc(C(C)C)c1N2C=CN(C2= [Au]N(S(=O)(=O)C(F)(F)F)S(=O) (=O)C(F)(F)F)c3c(cccc3C(C)C)C(C)C & 
 O=S(=O)(N([Au]=C1N(C=CN1C=2C(= CC=CC2C(C)C)C(C)C)C=3C(=CC=CC 3C(C)C)C(C)C)S(=O)(=O)C(F)(F)F)C(F)(F)F & 
 \includegraphics[width=2.1cm]{imagenes/sigmaAldrich/IPrAuNTf2.png} & 
 \includegraphics[width=2.2cm]{imagenes/sciFinder/pdf/IPrAuNTf2.pdf} \\
\hline

% Compuesto 29
 [Fe].[CH]1[CH][CH][C]([CH]1)P (c2ccccc2)c3ccccc3.[CH]4[CH][CH] [C]([CH]4)P(c5ccccc5)c6ccccc6 & 
 C=1C=CC(=CC1)P(C=2C=CC=CC2) [C-]34[CH]5=[CH]6[CH]7=[CH]3[Fe+2] 6789\%10\%1154[CH]=\%12[CH]\%11= [CH]\%10[C-]9(P(C=\%13C=CC=CC\%13) C=\%14C=CC=CC\%14)[CH]\%128 & 
 \includegraphics[width=2.2cm]{imagenes/sigmaAldrich/DPPF.png} & 
 \includegraphics[width=2.2cm]{imagenes/sciFinder/pdf/DPPF.pdf} \\
\hline

% Compuesto 30
 [Fe]I.[C-]\#[O+].[C-]\#[O+]. [CH]1[CH][CH][CH][CH]1 & 
 O\#C[Fe+2]1234([I-])(C\#O)[CH]= 5[CH]4=[CH]3[CH-]2[CH]51 & 
 \includegraphics[width=2.2cm]{imagenes/sigmaAldrich/Dicarbonylcyclopentadienyliodoiron(II).png} & 
 \includegraphics[width=2.2cm]{imagenes/sciFinder/pdf/Dicarbonylcyclopentadienyliodoiron(II).pdf} \\
\hline

% Compuesto 31
 No se encontró el compuesto en Sigma-Aldrich & 
 C=1C=C[N]2=C(C1)C3=CC=CC=4C=5C= CC=C[N]5[Fe+2]672([C-]34)[C-]=8C(=CC=CC8C=9C=CC=C [N]96)C=\%10C=CC=C[N]\%107 & 
 & 
 \includegraphics[width=2.2cm]{imagenes/sciFinder/pdf/(OC-6-11)-Bis[2,6-di(2-pyridinyl-κN)phenyl-κC]iron.pdf} \\
\hline



\end{longtable}

\end{landscape}


% \includepdf[pages=-, offset=0 0,landscape=true,picturecommand*={\put (\LenToUnit{.05\paperwidth},20) {[1]};}]{Iteracion3/pdfs/planificacion/Planificacion_inicial_iteracion3_21Noviembre2021.pdf}