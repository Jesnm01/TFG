\chapter{Conclusiones y trabajos futuros}

Blah blah \textbf{introduccion}

\section{Conclusiones}

blah blah \textbf{comparar los objetivos iniciales con los resultados obtenidos y cómo he alcanzado dichos objetivos}

En la actualidad, como exponía \textbf{en la Sección \ref{estadoArte}, } \textbf{cada software }
Se pretende por tanto con este trabajo, hacer una propuesta con el objetivo de resolver los conflictos entre las distintas bases de datos, usando una nueva nomenclatura canónica para moléculas organometálicas y mejoras en su dibujado.
Se aspira que en algún momento los cambios planteados se agreguen al software oficial, se desplieguen en un futuro release, y con el tiempo se extienda su uso y sea útil para los que utilizan esta herramienta. De hecho, ya se ha contactado con los administradores del repositorio oficial de GitHub para una posible contribución, quedando a la espera de la revisión del código y una respuesta por su parte.


\section{Trabajos futuros}

un punto a mejorar muy bueno sería el tema de la estereoquímica en la representación de las moléculas. A priori puede no parecer muy importante que se dibuje una línea plana o con cierta geometría (triangular, tetraédrica, cuadrada plana, octaédrica, etc), pero en ciertas áreas de la medicina y la bioquímica que trabajan con encimas y pequeñas proteínas, un determinado fármaco según su geometría o isomería puede tener efectos completamente distintos.
blah balh