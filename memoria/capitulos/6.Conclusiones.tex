\chapter{Conclusiones y trabajos futuros}

Blah blah \textbf{introduccion}

\section{Conclusiones}

blah blah \textbf{comparar los objetivos iniciales con los resultados obtenidos y cómo he alcanzado dichos objetivos}

En la actualidad, como exponía \textbf{en la Sección \ref{estadoArte}, } \textbf{cada software }
Se pretende por tanto con este trabajo, hacer una propuesta con el objetivo de resolver los conflictos entre las distintas bases de datos, usando una nueva nomenclatura canónica para moléculas organometálicas y mejoras en su dibujado.
Se aspira que en algún momento los cambios planteados se agreguen al software oficial, se desplieguen en un futuro release, y con el tiempo se extienda su uso y sea útil para los que utilizan esta herramienta. \textbf{esto no se si quitarlo de aqui y simplemente decirlo en la presentacion} De hecho, ya se ha contactado con los administradores del repositorio oficial de GitHub para una posible contribución, quedando a la espera de la revisión del código y una respuesta por su parte.

En química, hay una serie de reglas establecidas y la mayoría de moléculas convencionales se ajustan a ellas. Existen también muchas excepciones y particularidades, macromoléculas, proteínas, o compuestos de coordinación y organometálicos, que se rigen por sus propias normas. En general, creemos se han alcanzado resultados satisfactorios para el conjunto de datos con el que se ha trabajado. Es un conjunto de datos relativamente pequeño, pero se ajusta bien al alcance de un proyecto de estas características. 

Realmente el problema que estamos tratando aqui no son los algoritmos en sí, si no que todo el mundo utilice el mismo algoritmo de generacion canonico. OpenSmiles: "Canonical SMILES should not be considered a universal, global identifier (such as a permanent name that spans the WWW). Two systems that produces a canonical SMILES may use different rules in their code, or the same system may be improved or have bugs fixed as time passes, thus changing the SMILES it produces. A Canonical SMILES is primarily useful in a single database, or a system of related databases or information, in which all molecules were created using a single canonicalizer."

\section{Trabajos futuros}

Extender el estudio y realizar pruebas con un conjunto de moléculas mayor, adaptando poco a poco el sistema de dibujado para que se ajuste a las excepciones.

un punto a mejorar muy bueno sería el tema de la estereoquímica en la representación de las moléculas. A priori puede no parecer muy importante que se dibuje una línea plana o con cierta geometría (triangular, tetraédrica, cuadrada plana, octaédrica, etc), pero en ciertas áreas de la medicina y la bioquímica que trabajan con encimas y pequeñas proteínas, un determinado fármaco según su geometría o isomería puede tener efectos completamente distintos.
blah balh

para la parte de canonizacion, ahora mismo, está pensado para organometálica simplemente, la cosa sería ampliar la canonización para abarcar otras áreas de la química (poner alguna palabras complicada de estas, estereoquimica, etc)