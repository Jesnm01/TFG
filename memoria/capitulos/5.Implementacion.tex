
\chapter{Implementación y resultados}

Aquí la idea es ir poniendo las pruebas que vaya haciendo de las moléculas, y lo que vaya descubriendo.

En otro apartado, explicar el sistema de canonizacion (y los cambios en el dibujado, esto no se si es mejor directamente en resultados, puesto que tendré que poner fotos de cómo ha quedado, y expliclarlo sin fotos es medio raro) al que he llegado y sus reglas. Ya en la seccion de experimentacion, expondré los resultados.

 - espaciado de los bonds aumentado para mayor claridad y separacion entre los atomos. Hace que no se vea todo tan pegado. Util para moleculas planas o sin estructuras especiales (cps, o geometrias)

 - Se ha puesto el foco /centrado los esfuerzos de mejora de dibujado en estrcuturas Cp, muy comunes en organometalica. Describir cómo se detectan los Cp (hablar de los problemas con los anillos SSSR, peculiaridades, casos especificos, problemas y proceso de implementacion (de pensado del algoritmo o de como funciona a rasgos generales sin entrar en tema de codigo), como 1º trabajé con moleculas con 1 solo Cp, y luego con varias, una vez tuviera los bloques (hablar por tanto de la necesidad de distinguir/detectar no solo las estrcuturas, sino cuantas había y cuando empieza y acaba cada una, y qué carbonos forman parte de cada Cp))
 Ilustrar todo esto con imagenes es bueno.

 - Canonizado: comentar no se si aquí o en el estado del arte el sistema de canonizado propio de openbabel. se queria idear un sistema canonico para que independientemente de la forma en la que se escriba el SMILES de la molécula, siempre obtuvieramos el mismo SMILES resultado. Para esto, es necesario definir una serie de reglas que den prioridad a segun que atomos (manteniendo ovbiamente la correspondecia de los enlaces entre ellos) para luego mostrarlos en ese orden. Describir esas reglas detalladamente y con ejemplos a ser posible (al menos 1 para ilustrar el caso).

Se ha intentado dar un enfoque al canonizado orientado a la representacion 2D, de manera que se visualicen juntas y se puedan identificar claramenta la correspondecia entre una porcion del SMILES con una parte del dibujo. (ilustrar esto con una imagen que me invente yo) Quizas las reglas de las ramas no estén fundamentadas en ningún principio químico, pero me parece útil estructurar al SMILES canónico en base a las ramificaciones para seguir ese orden. (que no salte entre vecinos distintos, sino que recorra primero todos los neigbours en orden y cuando se acaben, pase al siguiente bloque/rama)

Despues de desciribr el proceso de cada cosa, mostrar sus resultados
Para los resultados de la canonizacion tendré que hacer una tabla comparando ambas cadenas o algo asi

 En otro apartado describir el testing. La mayoria de tests que he realizado son tests funcionales completos. Hacer test de unidad para los metodos importantes es complicado ya que necesitan una serie de parámetros y variables que se van seteando/creando sobre la marcha en multitud de metodos previos.

