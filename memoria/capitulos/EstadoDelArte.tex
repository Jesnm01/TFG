\chapter{Estado del arte y fundamentos teóricos}

Quizas sea mejor mover esta seccion justo despues de la introduccion para seguir con la tematica de la motivacion, y ya luego me meto con la gestion y planificacion

Puedo hacer una revision de la literatura existente hasta dia de hoy sobre el tema
Usar SCOPUS para esto, con terminos tipo: "SMILES" "molecule" "organometalic" (juntarlos o separarlos segun vea)

Hablar por aqui de la organometalica, representacion de moléculas, Hablar mas extendido de SMILES, SELFIES, e INCHI; cosas de dibujado de moleculas (los paquetes que hay), )
No se si meterlo aqui o en otro apartado, el diagrama de clases

la historia de la humanidad está marcada por la búsqueda de materiales que mejoren su calidad de vida, y los metales han sido parte crucial de esos cambios 

La materia que forma los seres vivos tiene en su composición sustancias cuya base I principal es el carbono. El estudio de estos compuestos constituye una rama de la química llamada química orgánica. La abundancia del carbono en el planeta es relativamente pequeña: aproximadamente un 0,03 \%; sin embargo, da lugar a millones de sustancias diferentes, mientras que los compuestos inorgánicos son solo unos pocos miles. ¿Qué hace a este elemento tan especial? Su estructura singular, que le permite formar largas cadenas en las cuales una pequeña variación da lugar a un compuesto distinto al anterior.

\bigskip