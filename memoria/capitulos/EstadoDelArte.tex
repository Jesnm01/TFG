\chapter{Estado del arte y fundamentos teóricos}

Quizas sea mejor mover esta seccion justo despues de la introduccion para seguir con la tematica de la motivacion, y ya luego me meto con la gestion y planificacion

Puedo hacer una revision de la literatura existente hasta dia de hoy sobre el tema
Usar SCOPUS para esto, con terminos tipo: "SMILES" "molecule" "organometalic" (juntarlos o separarlos segun vea)

Hablar por aqui de la organometalica, representacion de moléculas, Hablar mas extendido de SMILES, SELFIES, e INCHI; cosas de dibujado de moleculas (los paquetes que hay), )
No se si meterlo aqui o en otro apartado, el diagrama de clases

la historia de la humanidad está marcada por la búsqueda de materiales que mejoren su calidad de vida, y los metales han sido parte crucial de esos cambios 

La materia que forma los seres vivos tiene en su composición sustancias cuya base I principal es el carbono. El estudio de estos compuestos constituye una rama de la química llamada química orgánica. La abundancia del carbono en el planeta es relativamente pequeña: aproximadamente un 0,03 \%; sin embargo, da lugar a millones de sustancias diferentes, mientras que los compuestos inorgánicos son solo unos pocos miles. ¿Qué hace a este elemento tan especial? Su estructura singular, que le permite formar largas cadenas en las cuales una pequeña variación da lugar a un compuesto distinto al anterior.


(del archivo de informeReuniones puedo ir sacando cosas para meter en el estado del arte) Openbabel como tal no soporta el dibujado en 3D de las moléculas, por lo que los únicos dibujos que puede hacer son en 2D. Openbabel tampoco soporta el uso de wedge ni hash bonds para el dibujado de moléculas en 2D con perspectiva (es curioso porque en el código sí que hay funciones dedicadas a esto, pero luego cuando le metes símbolos SMILES de @@ y demás, los medio ignora. Sigo ejemplos del tutorial de Daylight, pero no salen los mismos dibujos. Otros símbolos más dedicados a geometría que viene en Daylight o en OpenSMILES, tipo @SP, @TB, @OH, los ignora por completo). Pero sí es capaz de generar archivos .sdf con información 3D, que se pueden usar en otros softwares de dibujado específicos como Avogadro (y no están mal, mucho mejor que los 2D desde luego)

\bigskip