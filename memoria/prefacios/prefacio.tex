\chapter*{}
%\thispagestyle{empty}
%\cleardoublepage

%\thispagestyle{empty}


% \cleardoublepage
\thispagestyle{empty}

\begin{center}
{\large\bfseries Detección de errores en bases de datos químicas}\\
\end{center}
\begin{center}
Jesús Navarro Merino\\
\end{center}

%\vspace{0.7cm}
\noindent{\textbf{Palabras clave}: Química computacional, Química organometálica, Nomenclatura canónica, SMILES, Representación de moléculas, Cp, OpenBabel, Bases de datos, SciFinder.}\\

\vspace{0.7cm}
\noindent{\textbf{Resumen}}\\

El desarrollo de la informática y la aplicación de sus métodos al mundo de la química ha propiciado grandes avances en esta ciencia. Sin embargo, un campo poco explorado dentro de la química computacional es la organometálica, de manera que muchas de las herramientas existentes no están preparadas para trabajar con ella. En esta área, para los investigadores es importante codificar las moléculas en representaciones lineales como SMILES y usar las herramientas software, para entre otras cosas, representar los compuestos gráficamente y comprenderlos mejor. La heterogeneidad en las distintas bases de datos públicas a la hora de consultar un mismo compuesto entorpece el trabajo de los investigadores.

Tras un análisis de esas inconsistencias, este proyecto propone una nomenclatura canónica para moléculas organometálicas en donde se le dé la suficiente importancia al metal, de manera que quede colocado el primero en el SMILES canónico resultado. En química de coordinación y organometálica, las reacciones y los compuestos no se explican bien siguiendo la teoría del enlace de valencia. Por tanto, se presentan también algunos cambios en el sistema de representación 2D de la herramienta utilizada, para ilustrar de una manera más adecuada los enlaces metal-ligando más habituales en compuestos organometálicos, las estructuras de ciclopentadienilo (Cp). Tras la validación de los resultados con diversos tests de consistencia, se obtienen finalmente resultados satisfactorios.

\cleardoublepage


\thispagestyle{empty}


\begin{center}
{\large\bfseries Error detection in chemical databases}\\
\end{center}
\begin{center}
Jesús Navarro Merino\\
\end{center}

%\vspace{0.7cm}
\noindent{\textbf{Keywords}: Chemoinformatics, Organometallic chemistry, Canonicalization, SMILES, Molecule depiction, Cp, OpenBabel, Databases, SciFinder.}\\

\vspace{0.7cm}
\noindent{\textbf{Abstract}}\\

The development of computer science and the application of its methods to the chemical world has led to great advances in this science. However, a less explored field within computational chemistry is organometallics, so many of the existing tools are not prepared to work with it. In this area, it is important for researchers to encode molecules in linear notations such as SMILES and use software tools to, among other things, depict compounds graphically and have a better understanding of them. Heterogeneity in various public databases when querying the same compound hinders scientists' work.

After an analysis of these inconsistencies, this project proposes a canonical nomenclature for organometallic molecules in which adequate importance is given to the metal, so that it is placed first in the resulting canonical SMILES. In coordination and organometallic chemistry, reactions and compounds do not fit well into the valence bond theory. Therefore, some changes are also presented in the 2D representation system of the tool used, to illustrate in a more accurate way the most common metal-ligand bonds in organometallic compounds, cyclopentadienyl (Cp) complexes. After validation of the results with several consistency tests, satisfactory outcomes are finally obtained.

\chapter*{}
\thispagestyle{empty}

\noindent\rule[-1ex]{\textwidth}{2pt}\\[4.5ex]

Yo, \textbf{Jesús Navarro Merino}, alumno de la titulación Grado en Ingeniría Informática de la \textbf{Escuela Técnica Superior
de Ingenierías Informática y de Telecomunicación de la Universidad de Granada}, con DNI 15429457E, autorizo la
ubicación de la siguiente copia de mi Trabajo Fin de Grado en la biblioteca del centro para que pueda ser
consultada por las personas que lo deseen.

\vspace{6cm}

\noindent Fdo: Jesús Navarro Merino

\vspace{2cm}

\begin{flushright}
Granada a \textbf{X} de julio de 2023.
\end{flushright}


\chapter*{}
\thispagestyle{empty}

\noindent\rule[-1ex]{\textwidth}{2pt}\\[4.5ex]

Dña. \textbf{Rocío Celeste Romero Zaliz}, Profesora del Departamento de Ciencias de la Computación e Inteligencia Artificial de la Universidad de Granada.


\vspace{0.5cm}

\textbf{Informan:}

\vspace{0.5cm}

Que el presente trabajo, titulado \textit{\textbf{Detección de errores en bases de datos químicas}},
ha sido realizado bajo su supervisión por \textbf{Jesús Navarro Merino}, y autorizamos la defensa de dicho trabajo ante el tribunal que corresponda.

\vspace{0.5cm}

Y para que conste, expiden y firman el presente informe en Granada a \textbf{X} de julio de 2023.

\vspace{1cm}

\textbf{La directora:}

\vspace{5cm}

\noindent \textbf{Rocío Celeste Romero Zaliz}

\chapter*{Agradecimientos}
\thispagestyle{empty}

       \vspace{1cm}


A mi tutora Rocío, por su esfuerzo y guía a lo largo del proyecto. Por su paciencia y ánimos en los momentos más críticos.
\\

A mis amigos, que por los buenos momentos vividos y el día a día con ellos, han hecho que estos años en la universidad hayan pasado casi sin darme cuenta.
\\

Y a mis padres, por su apoyo incondicional y confianza durante todo este trayecto. 