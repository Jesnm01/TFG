\documentclass[a4paper,11pt]{book}
%\documentclass[a4paper,twoside,11pt,titlepage]{book}
\usepackage{listings}
\usepackage[utf8]{inputenc}
\usepackage[spanish]{babel}
\usepackage{float}

% \usepackage[style=list, number=none]{glossary} %
%\usepackage{titlesec}
%\usepackage{pailatino}

\decimalpoint
\usepackage{dcolumn}
\newcolumntype{.}{D{.}{\esperiod}{-1}}
\makeatletter
\addto\shorthandsspanish{\let\esperiod\es@period@code}
\makeatother


%\usepackage[chapter]{algorithm}
\RequirePackage{verbatim}
%\RequirePackage[Glenn]{fncychap}
\usepackage{fancyhdr}
\usepackage{graphicx}
\usepackage{afterpage}
\usepackage{array}
\usepackage{makecell}
\usepackage{tabularray}
\usepackage{vcell}
\usepackage{booktabs}
\usepackage{textgreek}
\usepackage{amsmath} %Para poder usar \being{equation*}
\usepackage[gen]{eurosym} %Para usar el simbolo del euro en entornos varios 

\usepackage{longtable}

\usepackage[pdfborder={000}]{hyperref} %referencia

\usepackage[backend=biber, style=numeric, sorting=ynt]{biblatex}


% \usepackage{booktabs}
% \usepackage{dirtree}
% \usepackage{algorithmicx}
% \usepackage{algorithm}
\usepackage[font=small, margin=0.1cm]{caption}
% \usepackage{subcaption}
\usepackage{multirow}
% \usepackage{enumitem}
% \setlist{nosep}

\newcommand{\INDSTATE}[1][1]{\STATE\hspace{#1\algorithmicindent}}

% \usepackage[
%     a4paper,
%     left=2.8cm,
%     right=2.7cm,
%     top=4cm,
%     bottom=3cm
% ]{geometry}

% ********************************************************************
% Re-usable information
% ********************************************************************
\newcommand{\myTitle}{Detección de errores en bases de datos químicas\xspace}
\newcommand{\myDegree}{Grado en Ingeniería Informática\xspace}
\newcommand{\myName}{Jesús Navarro Merino\xspace}
\newcommand{\myProf}{Rocío Celeste Romero Zaliz\xspace}
%\newcommand{\mySupervisor}{Put name here\xspace}
\newcommand{\myFaculty}{Escuela Técnica Superior de Ingenierías Informática y de
Telecomunicación\xspace}
\newcommand{\myFacultyShort}{E.T.S. de Ingenierías Informática y de
Telecomunicación\xspace}
\newcommand{\myDepartment}{Departamento de ...\xspace}
\newcommand{\myUni}{\protect{Universidad de Granada}\xspace}
\newcommand{\myLocation}{Granada\xspace}
\newcommand{\myTime}{\today\xspace}
\newcommand{\myVersion}{Version 0.1\xspace}


\hypersetup{
pdfauthor = {\myName (email (en) ugr (punto) es)},
pdftitle = {\myTitle},
pdfsubject = {},
pdfkeywords = {palabra_clave1, palabra_clave2, palabra_clave3, ...},
pdfcreator = {LaTeX con el paquete ....},
pdfproducer = {pdflatex}
}

%\hyphenation{}


%\usepackage{doxygen/doxygen}
%\usepackage{pdfpages}
\usepackage{url}
\usepackage{color,colortbl,longtable}
\usepackage{tabularx}
\usepackage{lscape}
\usepackage[stable]{footmisc}
%\usepackage{index}

%\makeindex
%\usepackage[style=long, cols=2,border=plain,toc=true,number=none]{glossary}
% \makeglossary

% Definición de comandos que me son tiles:
%\renewcommand{\indexname}{Índice alfabético}
%\renewcommand{\glossaryname}{Glosario}

\pagestyle{fancy}
\fancyhf{}
\fancyhead[LO]{\leftmark}
\fancyhead[RE]{\rightmark}
\fancyhead[RO,LE]{\textbf{\thepage}}
\renewcommand{\chaptermark}[1]{\markboth{\textbf{#1}}{}}
\renewcommand{\sectionmark}[1]{\markright{\textbf{\thesection. #1}}}

\setlength{\headheight}{1.5\headheight}

\newcommand{\HRule}{\rule{\linewidth}{0.5mm}}
%Definimos los tipos teorema, ejemplo y definición podremos usar estos tipos
%simplemente poniendo \begin{teorema} \end{teorema} ...
\newtheorem{teorema}{Teorema}[chapter]
\newtheorem{ejemplo}{Ejemplo}[chapter]
\newtheorem{definicion}{Definición}[chapter]

\definecolor{gray97}{gray}{.97}
\definecolor{gray75}{gray}{.75}
\definecolor{gray45}{gray}{.45}
\definecolor{gray30}{gray}{.94}

\lstset{ frame=Ltb,
     framerule=0.5pt,
     aboveskip=0.5cm,
     framextopmargin=3pt,
     framexbottommargin=3pt,
     framexleftmargin=0.1cm,
     framesep=0pt,
     rulesep=.4pt,
     backgroundcolor=\color{gray97},
     rulesepcolor=\color{black},
     %
     stringstyle=\ttfamily,
     showstringspaces = false,
     basicstyle=\scriptsize\ttfamily,
     commentstyle=\color{gray45},
     keywordstyle=\bfseries,
     %
     numbers=left,
     numbersep=6pt,
     numberstyle=\tiny,
     numberfirstline = false,
     breaklines=true,
   }
 
% minimizar fragmentado de listados
\lstnewenvironment{listing}[1][]
   {\lstset{#1}\pagebreak[0]}{\pagebreak[0]}

\lstdefinestyle{CodigoC}
   {
	basicstyle=\scriptsize,
	frame=single,
	language=C,
	numbers=left
   }
\lstdefinestyle{CodigoC++}
   {
	basicstyle=\small,
	frame=single,
	backgroundcolor=\color{gray30},
	language=C++,
	numbers=left
   }

 
\lstdefinestyle{Consola}
   {basicstyle=\scriptsize\bf\ttfamily,
    backgroundcolor=\color{gray30},
    frame=single,
    numbers=none
   }


\newcommand{\bigrule}{\titlerule[0.5mm]}


%Para conseguir que en las páginas en blanco no ponga cabecerass
\makeatletter
\def\clearpage{%
  \ifvmode
    \ifnum \@dbltopnum =\m@ne
      \ifdim \pagetotal <\topskip
        \hbox{}
      \fi
    \fi
  \fi
  \newpage
  \thispagestyle{empty}
  \write\m@ne{}
  \vbox{}
  \penalty -\@Mi
}
\makeatother

\usepackage{pdfpages}


\addbibresource{bibliografia/bibliografia.bib}
% \addbibresource{sample.bib}
% \addbibresource{bibliografia/webs.bib}

\begin{document}
\begin{titlepage}
 
 
\newlength{\centeroffset}
\setlength{\centeroffset}{-0.5\oddsidemargin}
\addtolength{\centeroffset}{0.5\evensidemargin}
\thispagestyle{empty}

\noindent\hspace*{\centeroffset}\begin{minipage}{\textwidth}

\centering
\includegraphics[width=0.9\textwidth]{imagenes/logos/logo_ugr.jpg}\\[1.4cm]

\textsc{ \Large TRABAJO FIN DE GRADO\\[0.2cm]}
\textsc{ INGENIERÍA INFORMÁTICA}\\[1cm]
% Upper part of the page
% 
% Title
\noindent\rule[-1ex]{\textwidth}{2pt}\\[3ex]
{\Huge\bfseries Detección de errores en bases de datos químicas\\
}
\noindent\rule[-3ex]{\textwidth}{2pt}\\[3ex]
\end{minipage}

\vspace{1.5cm}
\noindent\hspace*{\centeroffset}\begin{minipage}{\textwidth}
\centering

\textbf{Autor}\\ {Jesús Navarro Merino}\\[2.5ex]
\textbf{Directora}\\
{Rocío Celeste Romero Zaliz}\\[2cm]
\includegraphics[width=0.3\textwidth]{imagenes/logos/etsiit_logo.png}\\[0.1cm]
\textsc{Escuela Técnica Superior de Ingenierías Informática y de Telecomunicación}\\
\textsc{---}\\
Granada, a 11 de julio de 2023
\end{minipage}
%\addtolength{\textwidth}{\centeroffset}
%\vspace{\stretch{2}}
\end{titlepage}


\chapter*{}
%\thispagestyle{empty}
%\cleardoublepage

%\thispagestyle{empty}


% \cleardoublepage
\thispagestyle{empty}

\begin{center}
{\large\bfseries Detección de errores en bases de datos químicas}\\
\end{center}
\begin{center}
Jesús Navarro Merino\\
\end{center}

%\vspace{0.7cm}
\noindent{\textbf{Palabras clave}: Química computacional, Química organometálica, Nomenclatura canónica, Cp, OpenBabel, SMILES, Bases de datos, SciFinder ......}\\

\vspace{0.7cm}
\noindent{\textbf{Resumen}}\\

Poner aquí el resumen.
\cleardoublepage


\thispagestyle{empty}


\begin{center}
{\large\bfseries Error detection in chemical databases}\\
\end{center}
\begin{center}
Jesús Navarro Merino\\
\end{center}

%\vspace{0.7cm}
\noindent{\textbf{Keywords}: Chemoinformatics, Organometallic chemistry, Canonicalization, Cp, SMILES, OpenBabel, Databases, SciFinder ....}\\

\vspace{0.7cm}
\noindent{\textbf{Abstract}}\\

Write here the abstract in English.

\chapter*{}
\thispagestyle{empty}

\noindent\rule[-1ex]{\textwidth}{2pt}\\[4.5ex]

Yo, \textbf{Jesús Navarro Merino}, alumno de la titulación Grado en Ingeniría Informática de la \textbf{Escuela Técnica Superior
de Ingenierías Informática y de Telecomunicación de la Universidad de Granada}, con DNI 15429457E, autorizo la
ubicación de la siguiente copia de mi Trabajo Fin de Grado en la biblioteca del centro para que pueda ser
consultada por las personas que lo deseen.

\vspace{6cm}

\noindent Fdo: Jesús Navarro Merino

\vspace{2cm}

\begin{flushright}
Granada a \textbf{X} de julio de 2023.
\end{flushright}


\chapter*{}
\thispagestyle{empty}

\noindent\rule[-1ex]{\textwidth}{2pt}\\[4.5ex]

Dña. \textbf{Rocío Celeste Romero Zaliz}, Profesora del Departamento de Ciencias de la Computación e Inteligencia Artificial de la Universidad de Granada.


\vspace{0.5cm}

\textbf{Informan:}

\vspace{0.5cm}

Que el presente trabajo, titulado \textit{\textbf{Detección de errores en bases de datos químicas}},
ha sido realizado bajo su supervisión por \textbf{Jesús Navarro Merino}, y autorizamos la defensa de dicho trabajo ante el tribunal que corresponda.

\vspace{0.5cm}

Y para que conste, expiden y firman el presente informe en Granada a \textbf{X} de julio de 2023 .

\vspace{1cm}

\textbf{La directora:}

\vspace{5cm}

\noindent \textbf{Rocío Celeste Romero Zaliz}

\chapter*{Agradecimientos}
\thispagestyle{empty}

       \vspace{1cm}


Poner aquí agradecimientos...


\frontmatter
\tableofcontents
\listoffigures
\renewcommand{\listtablename}{Índice de tablas}
\renewcommand{\tablename}{Tabla}
\listoftables

\mainmatter
\setlength{\parskip}{5pt}

% --------------------------------------Capítulos-----------------------------
\chapter{Introducción}




% \begin{tblr}{|Q[h,t]|Q[c,t]|Q[r,b]|}
% \hline
% {C[Au].c1ccc(cc1)P(c2ccccc2)c3ccccc3 \\ Left Left} & Middle Center & {\includegraphics{imagenes/SciFinder/Chloro[(1,1-biphenyl-2-yl)di-tert-butylphosphine]gold(I).png}} \\
% \hline
% \end{tblr}

La Química estudia la composición y estructura de la materia, sus propiedades y transformaciones. Estudia las sustancias, la energía y sus cambios durante las reacciones. Desde que se tienen registros, la química ha sido fundamental para el desarrollo de la humanidad, ya que ha permitido la producción de materiales, alimentos, medicamentos y energía, entre otros. Esto ha sido un proceso lento y exhaustivo a través de la experimentación. Por ejemplo, en 1881, Beilstein publica su Manual de Química Orgánica, que recogía 15000 compuestos orgánicos con sus propiedades \cite{handbook_1881}. Conforme la química se iba expandiendo, también lo hacía el volumen de datos que se generaban, siendo cada vez mas frecuentes preguntas como "¿alguien habrá sintetizado ya este compuesto?" \cite{applied_chemo_intro}

Eventualmente, hace unas cuantas décadas, se pensó que la cantidad de información que cada químico por su cuenta había acumulado, se podía compartir y hacer accesible a la comunidad científica a través de su almacenamiento en bases de datos \cite{chemo_a_textbook}. Con el desarrollo de técnicas de manipulación y tratamiento de esos datos surgió el término \emph{chemoinformatics} (en español, quimioinformática).

Las \emph{chemoinformatics} han cobrado gran importancia en los últimos años debido al aumento exponencial de datos experimentales generados en la investigación biomédica y química, y a la necesidad de manejar y analizar esta información de manera eficiente.
Esta disciplina ha sido influenciada por diversas áreas, como la química, matemáticas, estadística, biología y ciencias de la computación entre otras. Al parecer, su origen se remonta a la década de 1940, habiendo ya algunas investigaciones en el área, pero el término 'chemoinformatics' se lleva utilizado desde 1998 \cite{leach_introduction_2007}. Como tal, aun no hay un acuerdo en cuanto a su definición, seguramente por su carácter interdisciplinar, ni si quiera en cómo deletrearlo, pudiendo aparecer también como \emph{cheminformatics}, \emph{chemical informatics}, \emph{chemi-informatics}, y \emph{molecular informatics} entre otras \cite{leach_introduction_2007, brown_chemoinformaticsintroduction_2009}. En la literatura se discuten varias interpretaciones sobre su definición, unas más precisas y otras más generales: \cite{leach_introduction_2007, basic_overview_chemo, brown_chemoinformaticsintroduction_2009, chemo_a_textbook}


\begin{center}
\small
\textit{La mezcla de recursos de información para transformar datos en información, y la información en conocimiento, con el fin de tomar decisiones más rápidas y efectivas en la identificación y optimización de fármacos} [Brown 1998]
\end{center}

\begin{center}
\small
\textit{Chem(o)informatics es un término genérico que abarca el diseño, la creación, la organización, la gestión, la recuperación, el análisis, la difusión, la visualización y el uso de la información química. }[G. Paris 1999]
\end{center}

\begin{center}
\small
\textit{La aplicación de métodos informáticos para resolver problemas de química} [J. Gasteiger and T. Engel 2006]
\end{center}


A pesar de ello, son a día de hoy un componente esencial en el descubrimiento de sustancias químicas; sin duda es un área en constante evolución y su importancia solo aumentará en los próximos años, tanto en el descubrimiento de fármacos —que es como originariamente surgió y donde más impacto tiene en la sociedad— como en otros campos de la química.


Una herramienta de vital importancia en este ámbito son los sistemas de representación lineal. Algunos formatos que usan este tipo de representación son Wiswesser Line Notation (WLN), Sybyl Line Notation (SLN), Representation of structure diagram arranged linearly (ROSDAL), Simplified Molecular-Input Line-Entry System (SMILES) o IUPAC Chemical Identifier (InChI). Surgieron a medida que la química y la tecnología computacional avanzaban, y nos permiten codificar moléculas para su análisis y almacenamiento en bases de datos. En el siglo XIX, se desarrollaron varias formas de representación visual de moléculas, como las fórmulas estructurales que permitieron a los químicos dibujar y visualizar moléculas de manera más efectiva. Sin embargo, estas formas de representación no son adecuadas para su uso en la computación, ya que no son fácilmente legibles para los programas informáticos. Nosotros, los humanos, cuando vemos una estructura molecular dibujada la entendemos directamente, obtenemos una visión global de los símbolos que representan los enlaces y la distribución espacial de los átomos que la componen, pero los computadores no tienen esa facilidad. Por ello, se desarrollaron sistemas de notación lineal que permitían describir de manera más precisa y eficiente la estructura molecular, trabajando con tipos de datos sencillos, usando por ejemplo cadenas de caracteres.


\section{Motivación}

Los formatos de representación lineal llevan siendo un tema de interés e investigación para los científicos desde mediados del siglo XIX, evolucionando poco a poco y desarrollándose nuevas representaciones en función de las necesidades —principalmente computacionales— del momento y las limitaciones que se iban descubriendo \cite{107_years_linear_notations}. En la actualidad las más usadas SMILES, InChI, y SELFIES \cite{SELFIES}. Como comenté antes, una forma muy potente de representar moléculas y compuestos químicos es mediante cadenas de caracteres, y de esto justamente se encargan las representaciones lineales: traducir una molécula, con sus átomos, enlaces entre ellos, ciclos y otras propiedades, en una cadena de caracteres que la represente, y que la máquina y los propios químicos puedan entender. Sin embargo, hay diferencias notables entre las representaciones, tanto en la sintaxis de las cadenas que se generan como en las aplicaciones que se le puede dar a cada una de ellas.


SMILES, ideada por David Weininger, sale a la luz en 1988 satisfaciendo con creces las necesidades de procesamiento de información química que había, desbancando a la representación estandarizada del momento, Wiswesser Line Notation (WLN). Desde ese entonces SMILES se convirtió —y sigue siendo a día de hoy— en el estándar de representación lineal, ya que permite describir estructuras moleculares de una forma sencilla en un formato fácil de leer, lo que ha hecho que sea una herramienta popular en la química computacional, siendo la más usada entre investigadores y químicos. Pese a esto, SMILES tiene dos grandes inconvenientes: una misma molécula puede escribirse con varias cadenas SMILES distintas válidas, es decir, tiene sinónimos (Figura \ref{fig:sinonimos_smiles}); y no es robusto ni sintáctica ni semánticamente. En este sentido se podría generar un string que no represente una molécula válida, como lo es \emph{CC(CCCC}, el cual tiene un paréntesis sin cerrar (lo que implica que no se delimita cuándo acaba la rama). O generar una molécula que no sea químicamente viable como \emph{CO=CC}, que muestra un átomo de oxígeno neutro formando tres enlaces (superando el límite de enlaces covalentes que un oxígeno neutro puede tener) \cite{SELFIES}.

\begin{figure}[H]
\centering
    \includegraphics[scale=0.4]{imagenes/intro/sinonimos.png}
    \caption{Distintas cadenas SMILES válidas para el 1-methyl-3-bromo-ciclohexeno. \textbf{(a)} Considera el ciclo como la rama principal y el bromo como ramificación. \textbf{(b)} Hace el recorrido que marca la flecha, dejando parte del ciclo como una ramificación. Imagen extraída de \cite{weininger_smiles_1988}}
    \label{fig:sinonimos_smiles}
\end{figure}

Esto tiene especial relevancia en el ámbito del Machine Learning (ML) o Aprendizaje Automático. Aunque se sale del alcance de este trabajo, uno de los grandes objetivos de la química computacional es la creación o diseño de nuevas moléculas. Se podrían crear modelos de ML, en particular de redes neuronales, capaces de generar moléculas ficticias válidas, para posteriormente ver sus propiedades, valorarlas energéticamente para ver cuán estables son, y estudiar su viabilidad en distintas aplicaciones, entre otras cosas. SMILES dificulta esta tarea, y por ello, aparece en 2020, SELFIES (SELF-referencIng Embedded Strings), una nueva representación lineal 100\% robusta, muy usada actualmente para modelos generativos (ver \cite{SELFIES, krenn_self_referencing_2020} para más detalles de cómo soluciona los problemas de robustez y otras características de la representación). SELFIES es relativamente reciente y continuamente está ampliando sus funcionalidades, mejorando en simplicidad y facilidad de uso \cite{selfies_recent_2023}. Aun así, no se termina de instaurar entre la comunidad investigadora. Por último, InChI es creado en 2013 por la IUPAC (International Union of Pure and Applied Chemistry) como un proyecto para estandarizar el proceso de búsqueda de estructuras moleculares entre distintas bases de datos. Esto es porque InChI (International Chemical Identifier) genera una cadena canónica única para cada molécula, de manera que cada molécula tiene una sola representación, y dicha representación solamente hace referencia a esa molécula. La principal desventaja radica en su sintaxis y su estructura jerárquica, haciéndola complicada de leer y utilizar por los humanos. Por esto mismo, no es la mejor opción para usar en modelos generativos, pues tiene una serie de reglas y normas gramaticales y aritméticas que son complejas de aplicar al generar moléculas a través de modelos de ML\cite{heller_inchi_2015}.


Por todo lo anterior, me centraré en la notación SMILES durante el desarrollo de este trabajo. Dicho esto, existen diversas bases de datos en química donde se recoge gran cantidad de información acerca de los compuestos. Entiéndase esto como una colección estructurada y organizada que contiene datos sobre compuestos químicos, sus propiedades y relaciones con otros compuestos. Se utilizan para almacenar y recuperar información sobre moléculas, sustancias, reacciones, propiedades fisicoquímicas, e incluso literatura científica relacionada. Mencionaré ahora las más importantes y las que serán objeto de interés. \emph{PubChem}, una base de datos abierta que sirve información a millones de usuarios en todo el mundo, desde investigadores y estudiantes hasta el público general. Recogen para cada compuesto, información sobre su estructura, representaciones 2D y 3D, identificadores, propiedades químicas y físicas, patentes, avisos de toxicidad, etc. \cite{pubchem_website} 


\begin{table}[h!]
\small
\centering
    \begin{tabular}{m{7cm}>{\centering\arraybackslash}m{4cm}}
        \hline
        \textbf{Código SMILES} & \textbf{Representación 2D} \\
        \hline
        C[Au].c1ccc(cc1)P(c2ccccc2)c3ccccc3 & \includegraphics[width=2.2cm]{imagenes/sigmaAldrich/Methyl(triphenylphosphine)gold(I)} \\ [0.8cm]
        \hline
        Cl[Pd]Cl.C1CC=CCCC=C1 & \includegraphics[width=2.2cm]{imagenes/sigmaAldrich/Dichloro(1,5-cyclooctadiene)palladium(II).png} \\ 
        \hline
        Cl[Au].CP(C)C & \includegraphics[width=2.2cm]{imagenes/sigmaAldrich/Chloro(trimethylphosphine)gold(I).png} \\ [0.8cm]
        \hline
         Cl[Au].CC(C)(C)P(c1ccccc1-c2ccccc2)C(C)(C)C & \includegraphics[width=2.2cm]{imagenes/sigmaAldrich/Chloro[(1,1-biphenyl-2-yl)di-tert-butylphosphine]gold(I).png} \\
        \hline
        [Fe]I.[C-]\#[O+].[C-]\#[O+].[CH]1[CH][CH][CH][CH]1 & \includegraphics[width=2.2cm]{imagenes/sigmaAldrich/Dicarbonylcyclopentadienyliodoiron(II).png} \\
        \hline
        % Br[Ni]Br.COCCOC & \includegraphics[width=2.2cm]{imagenes/sigmaAldrich/Nickel(II) bromide ethylene glycol dimethyl ether complex.png} \\
        % \hline
        \end{tabular}
    \caption{Códigos SMILES y sus representaciones visuales según Sigma-Aldrich}
    \label{tabla:tabla_peq_intro_sigmaAldrich}
\end{table}

\emph{SciFinder}, una herramienta de investigación muy potente que permite explorar las bases de datos de CAS (American Chemical Society) las cuales contienen literatura sobre Química y otras disciplinas afines como Física, Biomedicina, Geología, Ingeniería Química, etc. Incluye referencias bibliográficas y resúmenes de artículos, informes, y libros entre otras cosas. Permite realizar búsquedas por estructura, nombres de sustancias o identificadores, reacciones en la que participa dicha sustancia, artículos y publicaciones que nombren el compuesto en cuestión, e incluso proveedores de compra \cite{scifinder_website}. Para el uso de esta herramienta es necesario acceder mediante la red de una institución autorizada (en este caso trabajo mediante VPN de la UGR) y seguir los pasos para registrarte \footnote{Pasos para el registro en SciFinder \url{https://bibliotecaugr.libguides.com/scifinder_scholar}}. Y \emph{Sigma-Aldrich}, una compañía de ciencia, química y biotecnología que se dedica a la producción y venta de productos químicos, reactivos, equipos y materiales de laboratorio. Ofrece herramientas, servicios, artículos y una gran variedad de productos químicos que se utilizan en investigación, biofarmacéutica, e industria entre otros ámbitos \cite{sigma_aldrich_web}. A través de su página web se enfocan al comercio electrónico pudiendo buscar y comprar productos, compuestos orgánicos e inorgánicos, agentes reactivos, isótopos para síntesis químicas, proteínas, enzimas, etc. De cada producto muestra información relevante como la ficha de datos de seguridad, detalles de las propiedades físicas y químicas así como algunas representaciones lineales del compuesto y la representación molecular en 2D.





\begin{table}[h!]
\small
\centering
    \begin{tabular}{m{7cm}>{\centering\arraybackslash}m{4cm}}
        \hline
        \textbf{Código SMILES} & \textbf{Representación 2D} \\
        \hline
        [Au+]([CH3-])[P](C=1C=CC=CC1) (C=2C=CC=CC2)C=3C=CC=CC3 & \includegraphics[width=2.2cm]{imagenes/sciFinder/pdf/Methyl(triphenylphosphine)gold(I).pdf} \\
        \hline
        [Cl-][Pd+2]123([Cl-]) [CH]=4CC[CH]3=[CH]2CC[CH]41 & \includegraphics[width=2.2cm]{imagenes/sciFinder/pdf/Dichloro(1,5-cyclooctadiene)palladium(II).pdf} \\
        \hline
        [Cl-][Au+][P](C)(C)C & \includegraphics[width=2.2cm]{imagenes/sciFinder/pdf/Chloro(trimethylphosphine)gold(I).pdf} \\
        \hline
        [Cl-][Au+][P](C=1C=CC=CC1C=2C=CC=CC2) (C(C)(C)C)C(C)(C)C & \includegraphics[width=2.2cm]{imagenes/sciFinder/pdf/Chloro[(1,1-biphenyl-2-yl)di-tert-butylphosphine]gold(I).pdf} \\
        \hline
        O\#C[Fe+2]1234([I-])(C\#O)[CH]=5[CH]4 =[CH]3[CH-]2[CH]51 & \includegraphics[width=2.2cm]{imagenes/sciFinder/pdf/Dicarbonylcyclopentadienyliodoiron(II).pdf} \\
        \hline
        % [Br-][Ni+2]1([Br-])O(C)CCO1C & \includegraphics[width=2.2cm]{imagenes/sciFinder/Nickel(II) bromide ethylene glycol dimethyl ether complex.png} \\
        % \hline
        \end{tabular}
    \caption{Códigos SMILES y sus representaciones visuales según SciFinder}
    \label{tabla:tabla_peq_intro_sciFinder}
\end{table}


Desde la Universidad de Granada, la tutora de este TFG colabora con el grupo de investigación del ICIQ (Instituto Catalán de Investigación Química) liderado por la profesora Mónica H. Pérez-Temprano. Su foco de investigación gira en torno al entendimiento de transformaciones catalíticas en las que participan compuestos organometálicos, descubriendo y diseñando reacciones más eficientes basadas en catalizadores metálicos (para más detalle sobre el grupo de investigación y sus ámbitos de trabajo, ver su sitio web \cite{ICIQ}). En resumen, intentan desarrollar enfoques más sostenibles para la síntesis de moléculas orgánicas usando la química organometálica. Como tal, necesitan codificar correctamente una molécula de organometálica en pos de trabajar con ella adecuadamente y utilizar todas las herramientas, para, entre otras cosas, poder dibujarla y entenderla mejor.


Uno de los principales problemas que se detectan en este ámbito es la heterogeneidad en las distintas bases de datos para un mismo compuesto o molécula. Para ilustrar esto, presento las Tablas \ref{tabla:tabla_peq_intro_sigmaAldrich} y \ref{tabla:tabla_peq_intro_sciFinder}. Ambas tablas comparan las mismas moléculas, mostrando el código SMILES y la representación visual que ofrecen las bases de datos Sigma-Aldrich y SciFinder respectivamente. Vemos diferencias claras en el tratamiento de los ciclos aromáticos, la especificación de las cargas de los átomos y la posición de algunas ramificaciones. Utilizo un subconjunto de 5 moléculas pertenecientes a la organometálica, seleccionadas desde un conjunto de datos de 30 moléculas considerados de interés por los químicos del ICIQ (disponible para su consulta en GitHub). En el Apéndice \ref{apend:pagina_tabla_intro_grande}, se puede consultar una tabla comparativa con el set de moléculas al completo. 





\section{Objetivos}
Por tanto, el objetivo principal de este Trabajo Fin de Grado es mejorar las herramientas existentes para trabajar con chemoinformatics, adaptándolas a moléculas organometálicas. Para ello, se establecen los siguientes subobjetivos:
\begin{itemize}
    \item Analizar las distintas bases de datos químicas disponibles y evaluar similitudes y diferencias en el almacenado de moléculas organometálicas.
    \item Diseñar una metodología para representar de forma canónica una molécula organometálica en formato SMILES.
    \item Mejorar la visualización de moléculas organometálicas representadas en formato SMILES.
\end{itemize} 

\section{Estructura de la memoria}
\textbf{esperar a tenerla mas avanzada para completar esto}






% \begin{longtable}{|c|>{\centering\arraybackslash}b{3cm}|m{3cm}|c|c|}
% \caption{Título de la tabla} \\
% \hline
% \textbf{Columna 1} & \textbf{Columna 2} & \textbf{Columna 3} & \textbf{Columna 4} & \textbf{Columna 5} \\ \hline
% \endfirsthead

% \multicolumn{5}{c}%
% {{\bfseries \tablename\ \thetable{} -- Continuación de la tabla}} \\
% \hline
% \textbf{Columna 1} & \textbf{Columna 2} & \textbf{Columna 3} & \textbf{Columna 4} & \textbf{Columna 5} \\ \hline
% \endhead

% \hline \multicolumn{5}{r}{\textit{Continúa en la siguiente página}} \\
% \endfoot

% \hline
% \endlastfoot

% Fila 1, Columna 1 & [Au+]([CH3-])[P](C=1C=CC=CC1) (C=2C=CC=CC2)C=3C=CC=CC3  & Fila 1, Columna 3 & Fila 1, Columna 4 & \includegraphics[width=2.2cm]{imagenes/sciFinder/Chloro(trimethylphosphine)gold(I).png} \\ \midrule
% Fila 1, Columna 1 & Fila 1, Columna 2 & Fila 1, Columna 3 & Fila 1, Columna 4 & \includegraphics[width=2.2cm]{imagenes/sciFinder/Chloro(trimethylphosphine)gold(I).png} \\ \hline
% Fila 1, Columna 1 & Fila 1, Columna 2 & Fila 1, Columna 3 & Fila 1, Columna 4 & \includegraphics[width=2.2cm]{imagenes/sciFinder/Chloro(trimethylphosphine)gold(I).png} \\ \hline
% Fila 1, Columna 1 & Fila 1, Columna 2 & Fila 1, Columna 3 & Fila 1, Columna 4 & \includegraphics[width=2.2cm]{imagenes/sciFinder/Chloro(trimethylphosphine)gold(I).png} \\ \hline

% \end{longtable}


\chapter{Estado del arte y fundamentos teóricos}

Quizas sea mejor mover esta seccion justo despues de la introduccion para seguir con la tematica de la motivacion, y ya luego me meto con la gestion y planificacion

Puedo hacer una revision de la literatura existente hasta dia de hoy sobre el tema
Usar SCOPUS para esto, con terminos tipo: "SMILES" "molecule" "organometalic" (juntarlos o separarlos segun vea)

Hablar por aqui de la organometalica, cosas de dibujado de moleculas (los paquetes que hay), tutorial de SMILES, y representacion de moléculas)
No se si meterlo aqui o en otro apartado, el diagrama de clases

\chapter{Gestión y Planificación del proyecto}



\section{Metodología}


\section{Gestión de recursos}

\subsection{Recursos humanos}

\subsection{Recursos materiales}

\subsection{Recursos software}

\section{Gestión de costes}

\subsection{Coste de recursos humanos}
Esto quizás dejarlo para el final, cuando tenga la cantidad total de horas trabajadas (aunque podría hacerlo con las "300"horas que se supone que le tengo que dedicar)


\subsection{Coste hardware}

\subsection{Otros costes}

\subsection{Presupuesto final}


\section{Análisis de riesgos}






\chapter{Diseño e Implementación}

Aqui meter los diagramas de clases y todos los metodos que he ido añadiendo/modificando
En otro apartado, explicar el sistema de canonizacion (y los cambios en el dibujado, esto no se si es mejor directamente en resultados, puesto que tendré que poner fotos de cómo ha quedado, y expliclarlo sin fotos es medio raro) al que he llegado y sus reglas. Ya en la seccion de experimentacion, expondré los resultados.

O quizas mejor, dejar esta seccion solamente para el diseño. Y la que se llama ahora mismo 'experimentacion', llamarla 'Implementacion y resultados'

\chapter{Experimentación}

Aquí la idea es ir poniendo las pruebas que vaya haciendo de las moléculas, y lo que vaya descubriendo. Problablemente exponer aqui tb el método o reglas de canonizacion a las que llegue.

% \input{capitulos/Conclusiones.tex}

% El nocite* es para que muestre todos los elementos de la bibliografia, aunque no se hayan usado para citar nada en el documento (usando \cite{})
% \nocite{*}
\printbibliography[title={Bibliografía}]
% \printbibliography{sample.bib}
% \printbibliography[type=online, title={Otras fuentes}]

% \printbibheading
% \printbibliography{nottype=misc, heading=subbibliography, title={Citas}}
% \printbibliography{type=misc, heading=subbibliography, title={Otras fuentes}}


% \nocite{*}
% \bibliography{sample}
% \bibliographystyle{ieeetr}
% \addcontentsline{toc}{chapter}{Bibliografía}
% \bibliographystyle{miunsrturl}
%


% ----------------------------------Apéndices-----------------------------
% \appendix
% \chapter{Tabla comparativa del set completo de moléculas}

\label{apend:tabla_intro_grande}


\begin{landscape}
% \begin{table}
% \begin{tabularx}{24cm}{|X|X|X|X|X|X|X|}
%    \hline
%     & \textbf{IEEE} & \textbf{CODATA} & \textbf{ACM} & \textbf{Springer Verlag} & \textbf{ELSEVIER} & \textbf{IOS PRESS} \\ 
%     \hline
%     Journal & Journal Transactions on Knowledge and Data Engineering & Data Science Journal & Journal of Data and Information Quality & International Journal of Data Science & Computional Statistics and Data Analysis & Data Science Journal \\
%     \hline
%     Organisation bzw. Verlag & Organisation & Organisation & Organisation & Verlag & Verlag & Verlag \\
%     \hline
%     Mitglieder bzw. Mitarbeiter & 400.000 Mitglieder & - & 78.000 Mitglieder & 15.323 (2016) Mitarbeiter & 30.500 (2011) Mitarbeiter & - \\
%     \hline
%     Editoren & Editor-in-Chief Xuemin Lin Editors-in-Chief Lei Chen & Editor-in-Chief Sarah Callaghan & - & Editor-in-Chief Longbing Cao & Co-Editors A.M. Colubi E.J. Kontoghiorghes B.U. Park & Editors-in-Chief Michel Dumontier Tobias Kuhn \\
%     \hline
%     Links & \url{http://ieeexplore.ieee.org/xpl/RecentIssue.jsp?punumber=69} & \url{https://datascience.codata.org/} & \url{https://dl.acm.org/citation.cfm?id=J1191} & \url{http://www.springer.com/computer/database+management+\%26+information+retrieval/journal/41060} & \url{https://www.journals.elsevier.com/computational-statistics-and-data-analysis/} & \url{https://www.iospress.nl/journal/data-science/} \\
%     \hline
%     Erstausgabe Jahr & 1989 & 2002 & 2009 & 2016 & 1983 & 2017 \\
%     \hline
% \end{tabularx}
% \caption{Übersicht der Konkurrenten}
% \end{table}


% \begin{table}[h!]
% \small
% \centering
%     \begin{tabular}{m{3cm}>{\centering\arraybackslash}m{6cm}m{5cm}m{4cm}m{4cm}}
%         \hline
%         \textbf{Nombre} & \textbf{SMILES SA} & \textbf{SMILES SF} & \textbf{Imagen SA} & \textbf{Imagen SF} \\
%         \hline
%         & & [Au+]([CH3-])[P](C=1C=CC=CC1) (C=2C=CC=CC2)C=3C=CC=CC3 &  & \includegraphics[width=2.2cm]{imagenes/sciFinder/Methyl(triphenylphosphine)gold(I).png} \\
%         \hline
%         & & [Cl-][Pd+2]123([Cl-]) [CH]=4CC[CH]3=[CH]2CC[CH]41 & & \includegraphics[width=2.2cm]{imagenes/sciFinder/Dichloro(1,5-cyclooctadiene)palladium(II).png} \\
%         \hline
%         & & [Cl-][Au+][P](C)(C)C & & \includegraphics[width=2.2cm]{imagenes/sciFinder/Chloro(trimethylphosphine)gold(I).png} \\
%         \hline
%         & & [Cl-][Au+][P](C=1C=CC=CC1C=2C=CC=CC2) (C(C)(C)C)C(C)(C)C & & \includegraphics[width=2.2cm]{imagenes/sciFinder/Chloro[(1,1-biphenyl-2-yl)di-tert-butylphosphine]gold(I).png} \\
%         \hline
%         & & O\#C[Fe+2]1234([I-])(C\#O)[CH]=5[CH]4 =[CH]3[CH-]2[CH]51 & & \includegraphics[width=2.2cm]{imagenes/sciFinder/Dicarbonylcyclopentadienyliodoiron(II).png} \\
%         \hline




%         & & O\#C[Fe+2]1234([I-])(C\#O)[CH]=5[CH]4 =[CH]3[CH-]2[CH]51 & & \includegraphics[width=2.2cm]{imagenes/sciFinder/Dicarbonylcyclopentadienyliodoiron(II).png} \\
%         \hline
%         & & O\#C[Fe+2]1234([I-])(C\#O)[CH]=5[CH]4 =[CH]3[CH-]2[CH]51 & & \includegraphics[width=2.2cm]{imagenes/sciFinder/Dicarbonylcyclopentadienyliodoiron(II).png} \\
%         \hline
%         & & O\#C[Fe+2]1234([I-])(C\#O)[CH]=5[CH]4 =[CH]3[CH-]2[CH]51 & & \includegraphics[width=2.2cm]{imagenes/sciFinder/Dicarbonylcyclopentadienyliodoiron(II).png} \\
%         \hline
%         & & O\#C[Fe+2]1234([I-])(C\#O)[CH]=5[CH]4 =[CH]3[CH-]2[CH]51 & & \includegraphics[width=2.2cm]{imagenes/sciFinder/Dicarbonylcyclopentadienyliodoiron(II).png} \\
%         \hline
%         & & O\#C[Fe+2]1234([I-])(C\#O)[CH]=5[CH]4 =[CH]3[CH-]2[CH]51 & & \includegraphics[width=2.2cm]{imagenes/sciFinder/Dicarbonylcyclopentadienyliodoiron(II).png} \\
%         \hline

%         \end{tabular}
%     \caption{Tabla extendida para el set de datos de 30 moléculas. Contiene el nombre del compuesto (uno de los varios sinónimos según el formato de nomenclatura de la IUPAC), la cadena SMILES extraída de Sigma-Aldrich (SA), la cadena SMILES extraída de SciFinder (SF), y las imágenes de las respectivas bases de datos (SA y SF)}
%     \label{tabla:tabla_grande_intro_label}
% \end{table}




\begin{longtable}{m{7cm}m{8cm}cc}
\caption{Tabla extendida para el set de datos de 30 moléculas. Contiene la cadena SMILES extraída de Sigma-Aldrich (SA), la cadena SMILES extraída de SciFinder (SF), y las imágenes de las respectivas bases de datos (SA y SF)}\\
\hline
\textbf{SMILES SA} & \textbf{SMILES SF} & \textbf{Imagen SA} & \textbf{Imagen SF} \\ \hline
\endfirsthead

\multicolumn{4}{c}%
{{\bfseries \tablename\ \thetable{} -- Continuación de la tabla en la página siguiente}} \\
\hline
\textbf{SMILES SA} & \textbf{SMILES SF} & \textbf{Imagen SA} & \textbf{Imagen SF} \\ \hline
\endhead

\hline \multicolumn{4}{r}{{Continúa en la siguiente página}} \\
\endfoot

\hline
\endlastfoot

% Compuesto 2
 C[Au].c1ccc(cc1)P(c2ccccc2) c3ccccc3 & 
 [Au+]([CH3-])[P](C=1C=CC=CC1) (C=2C=CC=CC2)C=3C=CC=CC3 & 
 \includegraphics[width=2.2cm]{imagenes/sigmaAldrich/Methyl(triphenylphosphine)gold(I).png} & 
 \includegraphics[width=2.2cm]{imagenes/sciFinder/pdf/Methyl(triphenylphosphine)gold(I).pdf} \\
\hline

% Compuesto 3
 Br[Pd]Br.c1ccc(cc1) P(c2ccccc2)c3ccccc3.c4ccc(cc4) P(c5ccccc5)c6ccccc6 & 
 [Br-][Pd+2]([Br-])([P](C=1C= CC=CC1)(C=2C=CC=CC2) C=3C=CC=CC3)[P](C=4C=CC=CC4) (C=5C=CC= CC5)C=6C=CC=CC6 & 
 \includegraphics[width=2.2cm]{imagenes/placeholder.png} & 
 \includegraphics[width=2.2cm]{imagenes/sciFinder/pdf/trans-Dibromobis(triphenylphosphine)palladium(II).pdf} \\
\hline

% Compuesto 4
 Cl[Pd]Cl.C1CC=CCCC=C1 & 
 [Cl-][Pd+2]123([Cl-]) [CH]=4CC[CH]3=[CH]2CC[CH]41 & 
 \includegraphics[width=2.2cm]{imagenes/sigmaAldrich/Dichloro(1,5-cyclooctadiene)palladium(II).png} & 
 \includegraphics[width=2.2cm]{imagenes/sciFinder/pdf/Dichloro(1,5-cyclooctadiene)palladium(II).pdf} \\
\hline

% Compuesto 5
 C1C[C@@H]2C[C@H]1CC2PC3C [C@@H]4CC[C@H]3C4.CN(C)c5ccccc5-c6ccccc6[Pd]Cl & 
 [Cl-][Pd+2]1([C-]=2C=CC=CC2C=3C =CC=CC3[N]1(C)C)[PH] (C4CC5CCC4C5)C6CC7CCC6C7 & 
 \includegraphics[width=2.2cm]{imagenes/placeholder.png} & 
 \includegraphics[width=2.2cm]{imagenes/sciFinder/pdf/SK-CC 01A.pdf} \\
\hline

% Compuesto 6
 C\textbackslash C(=N/O)c1ccc(O)cc1[Pd]Cl .C\textbackslash C(=N/O)c2ccc(O)cc2[Pd]Cl & 
 OC=1C=CC=2C(=[N](O)[Pd+2]3 ([Cl-][Pd+2]4([Cl-]3)[C-]=5 C=C(O)C=CC5C(=[N]4O)C)[C-]2C1)C & 
 \includegraphics[width=2.2cm]{imagenes/placeholder.png} & 
 \includegraphics[width=2.2cm]{imagenes/sciFinder/pdf/Bis[µ-chloro[5-hydroxy-2-[1-(hydroxyimino)ethyl]phenyl]palladium].pdf} \\
\hline


% Compuesto 7
 No se encontró el compuesto en Sigma-Aldrich & 
 FC=1C(Cl)=C(F)[C-](=C(F)C1Cl)[Pd+2] ([I-])([As](C=2C=CC=CC2)(C=3C=CC=C C3)C=4C=CC=CC4)[As](C=5C=CC=CC5) (C=6C=CC=CC6)C=7C=CC=CC7 & 
 \includegraphics[width=2.2cm]{imagenes/placeholder.png} & 
 \includegraphics[width=2.5cm]{imagenes/sciFinder/pdf/(SP-4-3)-(3,5-Dichloro-2,4,6-trifluorophenyl)iodobis(triphenylarsine)palladium.pdf} \\
\hline


% Compuesto 8
 No se encontró el compuesto en Sigma-Aldrich & 
 O=S(=O)([NH-][Pd+4]12([F-])([C-]=3C=CC=CC3C(C)(C)[CH2-]1) [N]=4C=CC=CC4C=5C=CC=C[N] 52)C6=CC=C(C=C6)C & 
 \includegraphics[width=2.2cm]{imagenes/placeholder.png} & 
 \includegraphics[width=2.2cm]{imagenes/sciFinder/pdf/Palladium, (2,2-bipyridine-κN1,κN1)[(2,2-dimethyl-1,2-ethanediyl)-1,2-phenylene]fluoro(4-methylbenzenesulfonamidato-κN)-, (OC-6-35).pdf} \\
\hline


% Compuesto 9
 Br[Ni]Br.COCCOC & 
 [Br-][Ni+2]1([Br-])O(C)CCO1C & 
 \includegraphics[width=2.2cm]{imagenes/sigmaAldrich/Nickel(II) bromide ethylene glycol dimethyl ether complex.png} & 
 \includegraphics[width=2.2cm]{imagenes/sciFinder/pdf/Dibromo(1,2-dimethoxyethane)nickel(II).pdf} \\
\hline


% Compuesto 10
 Cl[Ru](Cl)(C\#[O])(C\#[O])([PH](c1ccccc1) (c2ccccc2)c3ccccc3)[PH](c4ccccc4) (c5ccccc5)c6ccccc6 & 
 O\#C[Ru+2]([Cl-])([Cl-])(C\#O)([P](C=1C=CC=CC1) (C=2C=CC=CC2)C=3C=CC=CC3)[P] (C=4C=CC=CC4)(C=5C=CC=CC5) C=6C=CC=CC6 & 
 \includegraphics[width=2.2cm]{imagenes/placeholder.png} & 
 \includegraphics[width=2.2cm]{imagenes/sciFinder/pdf/Bis(triphenylphosphine)ruthenium(II) dicarbonyl chloride.pdf} \\
\hline

% Compuesto 11
 \ldots & 
 \ldots & 
 \includegraphics[width=2.2cm]{imagenes/placeholder.png} & 
 \includegraphics[width=2.2cm]{imagenes/placeholder.png} \\
\hline


% Compuesto 12
 \ldots & 
 \ldots & 
 \includegraphics[width=2.2cm]{imagenes/placeholder.png} & 
 \includegraphics[width=2.2cm]{imagenes/placeholder.png} \\
\hline



% Compuesto 13
 \ldots & 
 \ldots & 
 \includegraphics[width=2.2cm]{imagenes/placeholder.png} & 
 \includegraphics[width=2.2cm]{imagenes/placeholder.png} \\
\hline



% Compuesto 14
 \ldots & 
 \ldots & 
 \includegraphics[width=2.2cm]{imagenes/placeholder.png} & 
 \includegraphics[width=2.2cm]{imagenes/placeholder.png} \\
\hline


\end{longtable}

\textbf{POR TEMRINAR}

\end{landscape}


% \includepdf[pages=-, offset=0 0,landscape=true,picturecommand*={\put (\LenToUnit{.05\paperwidth},20) {[1]};}]{Iteracion3/pdfs/planificacion/Planificacion_inicial_iteracion3_21Noviembre2021.pdf}



%%\input{apendices/paper/paper}
%\input{glosario/entradas_glosario}
% \addcontentsline{toc}{chapter}{Glosario}
% \printglossary
% \chapter*{}
\thispagestyle{empty}

\end{document}